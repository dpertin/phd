%&these

\titre{Code à Effacement Mojette\\
    pour le Stockage Distribué}

%\soustitre{pour le Stockage Distribué}

%\titre{Approche Géométrique pour la Conception de Codes à Effacement Appliqués
%    dans le Stockage Distribué}

\title{Mojette Erasure Code\\
    for Distributed Storage}

%\subtitle{for Distributed Storage}

%\title{Geometrical Approach to Design Erasure Codes for Distributed Storage}

\author{M.}{Dimitri}{Pertin}

\discipline{Informatique et applications}

\sectionCNU{27}

\institution{UN}

\doctoralschool{Sciences et technologies de l'information, et mathématiques}

\laboratory{Institut de Recherche en Communications et Cybernétique de Nantes
    (IRCCyN)}%\\
    %École polytechnique de l'université de Nantes (Polytech Nantes, membre du
    %réseau Polytech des écoles d'ingénieurs polytechniques universitaires)}

\thesisnumber{}

\date{22 avril 2016}

\reviewer{M.}
    {Pierre}
    {Duhamel}
    {Directeur de recherche CNRS}
    {L2S Centrale Supélec, Paris}

\reviewer{M.}
    {Martin}
    {Quinson}
    {Professeur des universités}
    {ENS, Rennes}

\president{M.}
    {Imants}
    {Svalbe}
    {\emph{Senior Lecturer}}
    {Université Monash, Melbourne, Australie}

\examiner{M.}
    {Jérôme}
    {Lacan}
    {Professeur}
    {ISAE-SUPAERO, Toulouse}

\supervisor{M.}
    {Nicolas}
    {Normand}
    {Maitre de conférences titulaire de l'HDR}
    {Université de Nantes}
 
\cosupervisor{M.}
    {Benoît}
    {Parrein}
    {Maitre de conférences titulaire de l'HDR}
    {Université de Nantes}   

\guest{M.}
    {Évenou}
    {Pierre}
    {Ingénieur}
    {Rozo Systems, Nantes}


\begin{document}

\begin{resume}
    % pas plus de 1700 caractères, espaces compris
    \footnotesize
    
Les systèmes de stockage distribués sont sujets à des défaillances inévitables
qui entraînent l'inaccessibilité temporaire, voire la perte définitive de blocs
de données. La solution classique consiste à distribuer des copies de ces
données sur différents supports de stockage, mais cela engendre un coût de
stockage important.
%
Le codage à effacement est une alternative qui permet de réduire
considérablement la quantité de redondance au prix d'une complexité
calculatoire engendrée par les opérations d'encodage et de décodage.

La transformée Mojette (une version discrète et exacte de la transformée
de \radon) est capable de représenter de manière redondante l'information,
et de reconstruire efficacement cette information. % /!\

La conception d'une version systématique du code à effacement Mojette est la
première contribution de nos travaux de recherche. Ce code a un rendement
quasi-optimal, et les algorithmes pour le mettre en œuvre sont de faible
complexité.

La seconde contribution est une nouvelle méthode distribuée pour ré-encoder de
nouveaux symboles de mots de code, sans avoir à reconstruire explicitement
l'information initiale. Cette technique permet de rétablir un niveau de
redondance voulu.

Réalisés en collaboration entre l'IRCCyN et la société Rozo Systems, ces
travaux de recherche s'intègrent dans le système de fichiers
distribué RozoFS développé par l'entreprise. En conséquence, une attention
particulière a été portée aux performances des mises en œuvre réalisées.


\end{resume}

\begin{motscles}
    Code à effacement, transformée Mojette, stockage distribué,
    tolérance aux pannes.
\end{motscles}

\begin{abstract}
    \footnotesize
    
Distributed storage systems face inevitable failures which entail temporary
unavailability, or even permanent losses of data blocks. The classical solution
is to distribute copies of this data among different storage supports, but a
significant storage cost is involved.
%
Erasure coding is an alternative that greatly reduces this amount of redundancy
at the cost of computational complexity induced by encoding and decoding
operations.

The Mojette transform (a discrete and exact version of the \radon transform) is
able to depict a redundant representation of data, and to efficiently
reconstruct it.

The design of a systematic version of the Mojette erasure code is the first
contribution of our research work. This code has an almost optimal rate, and
the devised algorithms used to implement it have a low complexity.

The second contribution is a new distributed method to re-encode new codeword
symbols, without having to explicitly reconstruct the initial information. This
technique can be used to restore a desired level of fault-tolerance.

Conducted in collaboration with IRCCyN lab and Rozo Systems Inc, our research
work is part of the distributed storage system RozoFS developed by the
company. As a consequence, particular attention has been paid to the
implementation performance.


\end{abstract}

\begin{keywords}
    Erasure code, Mojette transform, distributed storage, fault tolerance.
\end{keywords}

\maketitle

\chapter*{Remerciements}


Cette aventure a débuté à Polytech, lorsque ma curiosité m'a poussé à choisir
un mystérieux sujet de recherche intitulé : "\emph{Code à effacement ?}".
Aujourd'hui, j'ai compris que ce "?" représente toute l'étendue du sujet.
Difficile de réaliser le chemin parcouru jusqu'ici. Et pourtant si j'y suis
parvenu, c'est grâce aux personnes qui m'ont soutenu.

% imants
Je tiens particulièrement à remercier Imants \textsc{Svalbe} pour avoir suivi
mes travaux avec son enthousiasme légendaire, depuis ma soutenance de projet de
recherche, jusque dans son rôle de président de jury de thèse.
% duhamel, quinson
Merci également à Pierre \textsc{Duhamel} et Martin \textsc{Quinson} qui m'ont
fait l'honneur de rapporter cette thèse.
%Leurs analyses et
%remarques pertinentes ont contribué à améliorer ce manuscrit, et m'ont
%particulièrement aidé.
% lacan
Je remercie aussi chaleureusement Jérôme \textsc{Lacan} pour son accueil à
l'ISAE, et pour avoir suivi et enrichi mon parcours de son expertise des codes,
et de sa personne.

% evenou, normand et parrein
Cette thèse n'aurait pu exister sans l'aide précieuse de Pierre
\textsc{Évenou}, dont les idées et la vision m'ont beaucoup inspiré, et
surtout sans Nicolas \textsc{Normand} et Benoît \textsc{Parrein}, compères
enthousiastes et passionnés, qui ont su stimuler ma curiosité, me guider durant
ces nombreuses années, et qui ont assuré dans les hauts, comme dans les bas.
Merci pour ces moments que l'on a partagés, face au tableau, comme au bistrot.

% rozofs
Un grand merci à l'équipe de Rozo Systems, notamment à Didier \textsc{Féron},
Jean-Pierre \textsc{Monchanin} et Sylvain \textsc{David}, génies de la
programmation qui m'ont tant appris, ainsi que Louis \textsc{le Gouriellec} et
Christophe \textsc{de la Guérande} pour nos captivantes discussions.
% ivc, riton, 
Merci également à Aurore pour notre complicité, Lukáš pour nos virées
nocturnes, le man Alex pour nos galères, Josselin le stagiaire et son
yoyo, Floflo et ses fruits secs, Lulu pour sa bonne humeur, Romu pour sa
mauvaise humeur, JPeG et son remorqueur, Romain le surréaliste, et tous les
membres d'IVC pour la convivialité.

% cunche & nicta
Je n'aurais sûrement pas débuté de travail de recherche sans Mathieu
\textsc{Cunche} ni les membres du NICTA qui ont su éveiller ma curiosité pour
ce domaine.
% isae
Je pense également aux membres de l'ISAE pour leur accueil, et plus
particulièrement à Jonathan \textsc{Detchart} pour sa connaissance des codes et
de la programmation. J'ai aussi beaucoup appris grâce à nos échanges avec Suayb
\textsc{Arslan}, Andrew \textsc{Kingston} et Shekhar \textsc{Chandra}. Merci
également à José \textsc{Martinez} pour ses connaissances en \LaTeX et son aide
dans la mise en place de l'expérimentation utilisant OpenMP.

Mille pensées à mes amis, camarades de Polytech, membres du JMB et joueurs du
baby pour ces bons moments. À mes frères et mes parents pour m'avoir toujours
aidé.

% lucie
Enfin, une pensée particulière à ma charmante Lucie, pour ses relectures, son
soutien, et pour avoir fait partie des meilleurs moments durant ces trois
années.



\dominitoc
\tableofcontents

\newrefsegment

\chapter*{Introduction générale}
\markboth{Introduction générale}{}

\addstarredchapter{Introduction générale}

\input{chapters/introduction.tex}
%\input{chapters/introduction.md}

\part{Codes à effacement en géométrie discrète}

\input{chapters/introduction_part_1.tex}
%\input{chapters/conclusion_part_1.md}

\input{chapters/chapter01.tex}
%\input{chapters/chapter01.md}

\input{chapters/chapter02.tex}
%\input{chapters/chapter02.md}

\input{chapters/chapter03.tex}
%\input{chapters/chapter03.md}

\input{chapters/conclusion_part_1.tex}
%\input{chapters/conclusion_part_1.md}

\part{Application au stockage distribué}

\input{chapters/introduction_part_2.tex}
%\input{chapters/introduction_part_2.md}

\input{chapters/chapter04.tex}
%\input{chapters/chapter04.md}

\input{chapters/chapter05.tex}
%\input{chapters/chapter05.md}

\input{chapters/chapter06.tex}
%\input{chapters/chapter06.md}

\input{chapters/conclusion_part_2.tex}
%\input{chapters/conclusion_part_2.md}

\chapter{Conclusion générale et perspectives}

%\addstarredchapter{Conclusion générale et perspectives}

\input{chapters/conclusion.tex}
%\input{chapters/conclusion.md}

\endrefsegment

\chapter*{Communications}
\markboth{Communications}{}

\addstarredchapter{Communications}

\input{chapters/communications.tex}
%\input{chapters/communications.md}

\printbibliography[
    heading=bibintoc,
    segment=1
]

% segment=1

% \printindex

%\section*{Nombre de ref}
%Attention : \total{citenum}\ !!! \textbackslash{}o/

\backmatter

\end{document}

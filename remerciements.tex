
Curieux, j'ai débuté cette aventure en dernière année de Polytech en répondant
à un mystérieux sujet de recherche intitulé : "\emph{Code à effacement ?}".
Aujourd'hui je comprend que ce "?" désigne l'étendu des travaux à accomplir.
Difficile de réaliser le chemin parcouru jusqu'ici. Et pourtant si j'y suis
parvenu, c'est grâce aux personnes qui m'ont soutenues.

% imants
Je tiens particulièrement à remercier Imants \textsc{Svalbe} pour avoir suivi
mes travaux avec son enthousiasme légendaire, depuis ma soutenance de projet de
recherche, jusque dans son rôle de président de jury de thèse.
% duhamel, quinson
Je suis également honoré et reconnaissant à Pierre \textsc{Duhamel} et Martin
\textsc{Quinson} d'avoir accepté de rapporter cette thèse.
%Leurs analyses et
%remarques pertinentes ont contribué à améliorer ce manuscrit, et m'ont
%particulièrement aidé.
% lacan
Je souhaite aussi remercier chaleureusement Jérôme \textsc{Lacan}, expert des
codes, pour son accueil à l'ISAE, et pour avoir suivi et enrichi ces travaux
pendant trois ans.

% evenou, normand et parrein
Cette thèse n'aurait pu exister sans l'aide de Pierre \textsc{Évenou}, dont les
idées et la vision m'ont beaucoup inspiré, ni sans Nicolas \textsc{Normand}
et Benoît \textsc{Parrein}, compères enthousiastes et passionnés, qui ont
su galvaniser ma curiosité et me guider pendant bien des années. Je garde de
très bons souvenirs des moments que l'on a partagés devant un tableau, comme au
bistrot.

% rozofs
Un grand merci à l'équipe de Rozo Systems, notamment à Didier \textsc{Féron},
Jean-Pierre \textsc{Monchanin} et Sylvain \textsc{David}, génies de la
programmation qui m'ont tant appris, ainsi que Louis \textsc{le Gouriellec} et
Christophe \textsc{de la Guérande} pour nos échanges.
% ivc, riton, 
Merci également à Aurore pour notre complicité, Lukáš pour nos virées
nocturnes, Alex, dit le man, Josselin le stagiaire et son trick de yoyo, Floflo
et ses fruits secs, Lulu pour sa bonne humeur, Romu pour sa mauvaise humeur,
JPeG et son remorqueur, Romain le surréaliste, et aux autres membres d'IVC pour
la convivialité.

% cunche & nicta
Je n'aurais sûrement pas débuté des travaux de recherche sans Mathieu
\textsc{Cunche} et les membres du NICTA qui ont su éveiller ma curiosité pour
ce domaine, et instaurer une ambiance chaleureuse.
% isae
Merci également aux membres de l'ISAE pour leur accueil, et plus
particulièrement à Jonathan \textsc{Detchart} pour sa connaissance des codes et
de la programmation. J'ai beaucoup appris en discutant avec Suayb
\textsc{Arslan}, Andrew \textsc{Kingston} et Shakhar \textsc{Chandra}. Je
remercie aussi José \textsc{Martinez} pour ses connaissances en \LaTeX et son
aide dans la mise en place de l'expérimentation utilisant OpenMP.

Mille mercis à mes amis, frères de Polytech, membres du JMB et joueurs du
baby, pour ces bons moments, et surtout à mes parents pour m'avoir toujours
aidé.

% lucie
Enfin, une pensée particulière à ma charmante Lucie, avec qui j'ai partagé les
meilleurs moments de ces trois années.


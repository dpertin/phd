
Erasure code is a technique of information theory that computes data redundancy
in distributed storage systems. This redundancy can be used to recover data
when a part of it is unavailable (referred as erased data). This method
significantly reduces the amount of redundancy compared to plain data
replication, at the cost of extra complexity which slow down encoding and
decoding operations.
%
The Mojette transform is a discrete and exact version of the \radon transform.
This technique is able to compute a redundant representation of an information,
and to rebuild it efficiently with a linear complexity.
%
The first contribution of this research work deals with the design of a
systematic version of the Mojette erasure code. This design provides an
almost-optimal code rate, with a low complexity.
%
Realized in a collaboration between IRCCyN and the company Rozo Systems, this
research work are embedded in the distributed file system RozoFS developed by
the company. A particular attention has been paid to the implementation
performances.
%
The second contribution is about the design of a novel distributed method to
generate extra codeword symbols. This technique enables the recovery and ease
the flexibility of the system fault-tolerance.

